\section{Introducción}
En este trabajo estamos a cargo de distribuir los pedidos al supermercado de la flamante unicornio Gloppi Ya entre sus compradores, de manera de poder maximizar el beneficio obtenido. Cada pedido será asignado a uno de nuestros compradores, quien se encargará de armarlos, pagarlos y entregarlos a los enviadores.


Nuestro objetivo es hacer esta asignación eficientemente de manera de poder satisfacer a mas clientes con los recursos que tenemos. Para lograrlo, vamos a pensar a cada pedido como dos valores de interés, su tamaño $W_{i}$, y su beneficio $P_{i}$. Por lo tanto, el tamaño de un pedido consistirá en la suma de los artículos que tiene un pedido, ya que éste número representa el tiempo que le tomará a un comprador realizar la compra en el supermercado.
Por lo tanto, dados todos los pedidos para un horario en particular, nuestra tarea es ver como asignarlos a la flota de compradores para poder sacar el mayor beneficio posible.


Si nos abstraemos, podemos ver que este problema se reduce a un caso particular del Problema de Multiples Mochilas. Es decir, el problema de dados un conjunto de elementos con un tamaño y un beneficio, asignarlos a un conjunto de contenedores.


La resolución de éste problema es compleja, por lo que vamos a realizar una aproximación. En lugar de asignar los pedidos a todos los compradores, vamos a tomar compradores uno a uno, de manera sucesiva. Primero le asignaremos pedidos al primer comprador de la mejor manera posible. Luego, con los elementos que sobraron, hacemos lo mismo con el segundo, y así sucesivamente.
De esta manera, nuestro problema pasa a consistir en que, dado un conjunto de elementos que tienen un peso y un beneficio, asignarlos a un solo comprador de la manera más eficiente. Éste problema es conocido como el Problema de la Mochila.


Veamos algunos ejemplos:

Supongamos que nuestro comprador puede cargar con hasta 9 kilos. Los elementos que queremos distribuir son:
\begin{itemize}
\item Una bolsa de papas de 5kg, que vendemos por 100\$.
\item Una bolsa de cebollas de 4kg, que vendemos por 70\$.
\item Un kilo de carne, que vendemos por 230\$.
\item Una bolsa de 3kg de naranjas, que vendemos por 105\$.
\end{itemize}

Busquemos entonces la combinación de estos elementos de mayor beneficio,tal que nuestro comprador la pueda cargar. Podemos ver que podemos juntar los primeros dos elementos para llegar exactamente a 9kg, con un beneficio de 170\$. Sin embargo, vemos que la bolsa de cebollas pesa lo mismo que la carne junto a las naranjas, por lo que podríamos intercambiarlas. Esto nos dejaría con un nuevo conjunto de 9kg, pero ahora con un beneficio de 435\$. Podemos obtener algo mejor que esto? Sabemos que a esta solución no lo podemos agregar nada más, ya que nos pasaríamos del peso límite. También es fácil ver a ojo que los últimos dos elementos son los más caros y al mismo tiempo los más livianos, por lo que vamos a querer tenerlos en nuestra solución. Esto nos deja con 5kg libres y como ya vimos, la mejor opción para esos 5kg es utilizar la bolsa de papas, obteniendo un beneficio de 435\$.

En este informe, estudiaremos cuatro técnicas algorítmicas distintas para la resolución de éste problema, a saber:
\begin{itemize}
\item Fuerza Bruta
\item Backtracking
\item Meet in the Middle
\item Programación Dinámica
\end{itemize}

Luego de implementar los 4 algoritmos, veremos sus complejidades, cuán buena es su performance cuando se enfrentan a casos reales, y veremos cómo se comparan entre ellos.
\label{sec:introduccion}
